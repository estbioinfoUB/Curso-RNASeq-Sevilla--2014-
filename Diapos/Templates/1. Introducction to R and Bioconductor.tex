\documentclass{beamer}
\usepackage{hyperref}
\usepackage[utf8]{inputenc}
%\usepackage[T1]{fontenc}
\usepackage{amsmath,amssymb,amsfonts,textcomp,setspace,graphicx,tikz,color}
\usepackage[absolute,overlay]{textpos}
  \setlength{\TPHorizModule}{1mm}
  \setlength{\TPVertModule}{1mm}

%%%%%%%%%%%%%%%%%%%%%%%%%%%%%%%%%%%%%%%%%%%%%%%%%%%%%%%%%%%%%%
%%%%%%%%%%%%   MODIFICAR AQUESTA SECCIÓ      %%%%%%%%%%%%%%%%%
%%%%%%%%%%%%%%%%%%%%%%%%%%%%%%%%%%%%%%%%%%%%%%%%%%%%%%%%%%%%%%
\title{Introduction to R and Bioconductor}
\subtitle{Id 353 – Òscar Reig - IDIBAPS}
\author[Euclid]{Euclid of Alexandria \texttt{euclid@alexandria.edu}}
\institute[UEB]{Unitat d'Estadística i Bioinformàtica (UEB)\\
		Edifici Mediterrània (Lab 131) \\
		Vall d'Hebron Istitut de Recerca (VHIR)
		}
\date{\today}
%%%%%%%%%%%%%%%%%%%%%%%%%%%%%%%%%%%%%%%%%%%%%%%%%%%%%%%%%%%%%%
%%%%%%%%%%%%%%%%%%%%%%%%%%%%%%%%%%%%%%%%%%%%%%%%%%%%%%%%%%%%%%
%%%%%%%%%%%%%%%%%%%%%%%%%%%%%%%%%%%%%%%%%%%%%%%%%%%%%%%%%%%%%%
\usetheme{UEBTheme}
\newcommand{\cinlaw}{\buildrel  { \cal L } \over {\small \buildrel \longrightarrow  \over {n \rightarrow \infty}
}}
\newcommand{\Prof}{{\mathbf P}}
\newcommand{\rank}{{\rm rank}}
%itemize
\newcommand{\bit}{\begin{itemize}}
\newcommand{\eit}{\end{itemize}}

%enumerate
\newcommand{\ben}{\begin{enumerate}}
\newcommand{\een}{\end{enumerate}}

%Blue box
\newcommand{\bB}[1]{\begin{problock}{#1}}
\newcommand{\eB}{\end{problock}}

%Green box
\newcommand{\bG}[1]{\begin{exampleblock}{#1}}
\newcommand{\eG}{\end{exampleblock}}

%Red box
\newcommand{\bR}[1]{\begin{alertblock}{#1}}
\newcommand{\eR}{\end{alertblock}}

%colors
\definecolor{dB}{RGB}{80,13,138}
\definecolor{lB}{RGB}{153,0,153}

\definecolor{dG}{rgb}{0.00,0.50,0.00}
\definecolor{lG}{rgb}{0.71,0.81,0.69}

%custom blocks

\newenvironment<>{problock}[1]{%
  \begin{actionenv}#2%
      \def\insertblocktitle{#1}%
      \par%
      \mode<presentation>{%
       \setbeamercolor{block title}{fg=white,bg=Plum}
       \setbeamercolor{block body}{fg=black,bg=lightpurple}
       \setbeamercolor{itemize item}{fg=Plum}
       \setbeamertemplate{itemize item}[triangle]
     }%
      \usebeamertemplate{block begin}}
    {\par\usebeamertemplate{block end}\end{actionenv}}



\begin{document}

\begin{frame}[plain]
%\addtocounter{totalframenumber}{-1}
\titlepage
\end{frame}

\begin{frame}[Frame 1]
\addtocounter{framenumber}{-1}
\frametitle{Table of Contents}
\tableofcontents
\end{frame}

%%%%%%%%%%%%%%%%%%%%%%%%%%%%%%%%%%%%%%%%%%%%%%%%%%%%%%%%%%%%%%%%%%%%%%%%%
%%%%%%%%%%%%      MODIFICAR A PARTIR D'AQUI       %%%%%%%%%%%%%%%%%%%%%%%
%%%%%%%%%%%%%%%%%%%%%%%%%%%%%%%%%%%%%%%%%%%%%%%%%%%%%%%%%%%%%%%%%%%%%%%%%

\section{R}

\subsection{Objectives}

  \begin{frame}
   \frametitle{A bit of interaction?}
    %\bB{The \textbf{\emph{main objective}} is...}
     %   to find differentially expressed genes (DEG) associated with Inflammatory Bowel 
      %  Disease (IBD) between samples of different tissues from bowels infected with three 
       % types of bacteria and a reference control
    %\eB
    
    \bit
      \item What is your R knowledge, on a 0(beginner) to 2 (expert) scale?

	      			\item How deep is your knowledge with R packages related 
	      			to NGS, on a 0(none) to 2 (good)scale?
	      			\item What analyses do you plan to do in R?


    \eit
\end{frame}

\subsection{Specific Objectives}
  \begin{frame}
   \frametitle{What is R?}
	
    \ben
      \item an implementation of the S language \tiny{(Bell Laboratories,Rick Becker, John Chambers and Allan Wilks)}
      \normalsize {
      \item R is an integrated suite of software for}
      		\bit
      			\item data manipulation
      			\item calculation and
      			\item graphical display.
      		\eit
    \een
  \end{frame}



\subsection{Basic information of samples}

   \begin{frame}
   \frametitle{What is R?(c'ed)}
	
    \ben
      \item R is a vehicle for newly developing methods of interactive data analysis
 
      		\bit
      			\item devolops rapidly
      			\item is being extenden by a large collection of packages
			    \bit
				\item Comprehensive R Archive Network (CRAN)
				\item Bioconductor
			    \eit
      			
      		\eit
      	\item However, most programs written in R are essentially ephemeral, written for a single piece of data analysis
    \een
  \end{frame}
  
 

   \begin{frame}
   \frametitle{R specifics}
	
    \bit
      \item a suite of operators for calculations on arrays, in particular matrices
 
      \item an ``environment'':
      
      		\bit
      			\item a fully planned and coherent system
      			\item can be saved, loaded,exchanged  			
      		\eit
    \eit
  \end{frame}
  
  \begin{frame}
  \frametitle{R and statistics}
   \bit
      \item R is an environment
	  \bit
	    \item not designed for statistics
	    \item many classical and modern statistical techiniques implemented
	  \eit \item R is an environment
	  \bit
      \item Differnece with S,S-plus,SAS and SPSS
	    \item minimal ouput
	    \item minimal number of object
	  \eit
      \eit
     \end{frame}

   
     \begin{frame}
     \frametitle{R and the window system}
      \bit
	\item R comes with a graphical system on all plataform
	  \bit
	    \item console like: Unix
	    \item GUI and console: Mac, Windows
	  \eit
	 \item Integrated Devoeloper Interface (IDE) have been developed
	 \bit
	    \item StatET plugin ( \url{http://www.walware.de/goto/statet}) for eclipse
	    \item Rstudio (\url{http://rstudio.org})
	 \eit
      \eit
      
     \end{frame}
     
        \begin{frame}
     \frametitle{Using R interactively}
      \bit
	  \item R enviroment is vey similar to Unix
	    \bit
		\item Is command for listing,...
		\item The syntax is only slightly different:
		  \bit
		      \item ls () instead of ls
		  \eit
	    \eit
	   \item Documentation and help pages always avaliable:
	    \bit
		\item through the ``?'' command (perfect match)
		\item through the ``?'' command (fuzzy matching)
		\item through hel.start() if you have a windows system
		\item searchable through help.search()
	    \eit
	  
      \eit
     \end{frame}

          \begin{frame}
     \frametitle{CRAN}
      \bit
	  \item The comprehensive R Archive
	    \bit
		\item 5578 packages! (26 May 2014)
		\item easy to install
		  \bit
		      \item R CM INSTALL (cmd line)
		      \item install.packages (from within the environment
		  \eit
	    \eit
      \eit
     \end{frame}
     
     
  \section{Bioconductor}
  
  \begin{frame}
     \frametitle{Bioconductor: history and overview}
      \bit
	  \item Gentleman et al. Bioconductor: open software development for computational biology and bioinformatics.
	  Genome Biology (2004) vol. 5 (10) pp. R80
	    \bit
		\item Fred Hutchinson Cancer Research Center (FHCRC)
	    \eit
	   \item A set og packages developed for the analysis and comprehension of high throughput genetic data
	    \bit
		\item >1.100 packages (554 soft.,600 annot.)
		\item >300 developers, >4.000 citations
	    \eit
	   \item Focus on microarray at first, and on Next Generation Sequencing as of 2008.
      \eit
     \end{frame}
     
     
  
     \begin{frame}
     \frametitle{FHCRC,BIOC core packages}
      \bit
	  \item Input and Output
	    \bit
		\item rtacklayer,\textbf{Rsamtools,ShortRead}
	    \eit
	  \item Sequence manipulation
	    \bit
		\item \textbf{Biostrings}
	    \eit
	  \item Range-based manipulations:
	    \bit
		\item \textbf{IRanges,GenomicRanges}
	    \eit
	  \item Annotations
	    \bit
		\item \textbf{GenomicFeatures}, AnnotationDbi, BSgenome
	    \eit
      \eit
     \end{frame}
     
       \begin{frame}
     \frametitle{}
      \bit
	  \item DIBUIX!
      \eit
     \end{frame}
     
       
     
      \begin{frame}{titre}
      \frametitle{53 Contributed packages (Sep. 2012)}
  \begin{columns}%
  
  
    \begin{column}[t]{0.45\textwidth}%
     \begin{itemize}
     \item Chip-seq(14)
      \begin{itemize}
        \item BayesPeak, CSAR, ChIPpeakAnno, ChiPseqR, ChIPsim, PICS, chipseq,...\\   
      \end{itemize}
     \item RNA-seq(18)
       \begin{itemize}
       \item DEGseq, DESeq, Genominator, baySeq, edgeR, srnaSeqMao, goseq, gage, easyRNASeq,..
      \end{itemize}
      \end{itemize}
    \end{column}
    
   
    \begin{column}[t]{0.45\textwidth}%
      \begin{itemize}
      \item \textbf{Infrastructure}: genomeIntervals, girafe, cqn
        \item \textbf{base calling}: Rolexa
        \item \textbf{Visualization}: HilbertVis HilbertVisGUI
        \item \textbf{motif:} MotIV, rGADEM
        \item \textbf{domain-specific}: MEDIPS,OTUbase, R453Plus1Toolbox
        \item \textbf{database}: SRAdb, oneChannelGUI
        \item \textbf{smRNA}: segmentSeq
      \end{itemize}
    \end{column}
  \end{columns}
\end{frame}

\section{Outlook}
\begin{frame}
 \frametitle{Outlook}
  \bit
      \item Numerous packages,bioC or third-party
      \item Complex, evolving infrastructure
      \item Very active community
	\bit
	    \item an R release every year
	    \item a Bioc release every ~ 6 months
	    \item bioconductor website
	      \bit
		  \item \url{http://bioconductor.org}
	      \eit
	     \item bion mailing lists
	      \bit
		  \item bioc-sig-sequencing
		  \item bioc-devel (heavy traffic)
	      \eit
	      
	\eit
  \eit
\end{frame}

     


\end{document}